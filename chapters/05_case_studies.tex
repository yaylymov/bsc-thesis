% !TeX root = ../main.tex
% Add the above to each chapter to make compiling the PDF easier in some editors.

\chapter{Case Studies}\label{chapter:applications}
With the bond of art and technology, deepfakes are slowly redifing the state of 
entertainment. Their ability to transform audios and visuals offers creators better
possibilities to create new type of content. From refining the quality of amateur
videos to colorizing black and white movies, deepfakes are reshaping the entertainment
and art industries. 


\section{Entertainment and Art}
Deepfakes are popular in many creative areas. For example, a rapper Kendrick Lamar,
used deepfake in 2022 music video to take on looks of famous celebrities. In the 
renewed Star Wars series, deepfake technology was used to resurrect characters like
Princess Leia and Moff Tarkin, despite the original actors having passed away~\cite{motion-analysis}. 

The real question is: Is deepfake technology a blessing or a curse for the talent?
Of course it offers scalability. An actor can feature in global commercials or websites
without constant traveling or learning new languages. For instance, Synthesia\footnote{\url{https://www.synthesia.io/}}
did this with two commercials starring rapper Snoop Dogg. Insead of reshooting for a
rebranded commercial, they altered Snoop Dogg's mouth movements to match the new brand 
name using deepfakes~\cite{wipo-magazine}.

One of the positive implications of deepfakes in Arts industry is for example, the Salvador
Dalí Museum introduced \textit{Dalí Lives}, a digital revival of the deceased artist 
Salvador Dalí using deepfakes~\cite{salvador-dali, salvador-dali2}. 
This allows visitors to interact with the artist, hearing tales from his life, and 
even taking selfies. The Museum used an encoder-decoder deepfake technique, training encoders on Dalí's 
images and footage. An actor resembling Dalí was then mapped with Dalí's features using
decoders (\autoref{dali-youtube}). 

\begin{figure}[ht]
	\centering
	\includegraphics[width=0.61\columnwidth]{figures/dali}
	\caption{Screenshot taken from Dalí Lives~\cite{salvador-dali-youtube}.}\label{dali-youtube}
\end{figure}

Deepfakes, while being helpful and revolutionary, have notable weaknesses and can pose serious
threats. They have been misused for creating fake celebrity videos, commiting fraud, and manipulating
political content, leading California to ban making political deepfakes during election
season in 2019~\cite{salvador-dali,california}. 


\section{Politics and Media}
Deepfakes have had an influence on politics and media as well. On positive side of this context,
some political figures have used deepfakes in their campaigns for creative advertisements. 
For instance, during the 2020 Delhi Legislative Assembly election in India, a deepfake video 
of the president of India's \ac{BJP} party, Manoj Tiwari, spread on WhatsApp, as reported 
by Vice~\cite{vice}. In this first-ot-its-kind campaign use, the original video of Tiwari speaking 
English was changed to appear as if he spoke in Haryanvi, a Hindi dialect, targeting specific 
voters. The \ac{BJP} collaborated with The Ideaz Factory to produce such deepfakes, aiming to
reach India's diverse linguistic audience. This particular deepfake reached reportedly 15 million
people across WhatsApp groups~\cite{india}. 

However, the implications of deepfakes in politcs and
media are significat. The risk of spreading misinformation is high. There have been past 
occurrences where fake videos were used to damage the reputation of political individuals.
Regulating political deepfakes is complex. While potential laws could aim to ban manipulated
content of politicians, they'd need exceptions to safeguard artistic and satirical content, making
implementation of those laws challenging due to the nature of satire and art~\cite{politics,vanity-fair}. 
Promising solutions are emerging in tech industry, as tech giants like Adobe and Microsoft, alongside
startups like Truepic, are developing tools for authenticity verification.

Ultimately, promoting digital media awareness is esstial, encouraging everyone to be critical
of what they witness.