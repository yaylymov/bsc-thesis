% !TeX root = ../main.tex
% Add the above to each chapter to make compiling the PDF easier in some editors.

\chapter{Analysis of Publicly-Available Deepfake Tools}
For a thorough analysis in this study, a variety of tools were picked. Their selection 
was not arbitrary; instead, it was based on the clear criteria detailed in~\autoref{tab:selection_criteria}.
Tools were chosen with a focus on public availability, ensuring that everyone can access 
and benefit from them. Every tool in this study is open to the public, making the findings 
broadly applicable.

\begin{figure}[htbp]
    \centering
    \begin{tikzpicture}[
        box/.style={draw, rectangle, minimum height=2em, minimum width=3em, text centered},
        bluebox/.style={box, fill=lightblue, draw=darkblue},
        greenbox/.style={box, fill=lightgreen, draw=darkgreen},
        orangebox/.style={box, fill=lightpink, draw=darkpink},
        greenframebox/.style={box, draw=darkgreen},
        orangeframebox/.style={box, draw=darkpink},
        line/.style={draw, -latex}
    ]

    % Nodes
    \node[bluebox] (deepfakes) {Deepfakes Detection};
    \node[greenbox, below left=0.5cm and 1cm of deepfakes] (video) {Video};
    \node[orangebox, below right=0.5cm and 1cm of deepfakes] (image) {Image};

    \node[greenframebox, below left=1cm and 0.7cm of video] (video1) {Deepware};
    \node[greenframebox, below=1cm of video] (video2) {Seferbekov};
    \node[greenframebox, below right=1cm and 0.7cm of video] (video3) {NtechLab};

    \node[orangeframebox, below left=1cm and 0.7cm of image] (image1) {Facetorch};
    \node[orangeframebox, below=1cm of image] (image2) {Illuminarty};
    \node[orangeframebox, below right=1cm and 0.7cm of image] (image3) {AI or Not};

    % Paths
    \path[line, lightblue] (deepfakes) -- (video);
    \path[line, lightblue] (deepfakes) -- (image);

    \path[line, lightgreen] (video) -- (video1);
    \path[line, lightgreen] (video) -- (video2);
    \path[line, lightgreen] (video) -- (video3);

    \path[line, lightpink] (image) -- (image1);
    \path[line, lightpink] (image) -- (image2);
    \path[line, lightpink] (image) -- (image3);

    \end{tikzpicture}
    \caption{Categorization of deepfake detection tools}\label{fig:deepfake-tools}
\end{figure}

As depicted in~\autoref{fig:deepfake-tools}, three tools were chosen for video 
detection and another three for image detection. Every tool comes with its own 
strengths and weaknesses, ranging from how users can access it, the ease of 
installation, to understanding the results it produces. The selection aimed to 
cover a range of capabilities, as shown in~\autoref{tab:evaluation_metrics}, 
to ensure a comprehensive analysis.

\section{Deepware}
\section{Seferbekov}
\section{NtechLab}
\section{Facetorch}
\section{Illuminarty}
\section{AI or Not}
\section{Comparative Analysis}