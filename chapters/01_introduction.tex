% !TeX root = ../main.tex
% Add the above to each chapter to make compiling the PDF easier in some editors.

\chapter{Introduction}\label{chapter:introduction}
The rapid and continuous development of \ac{AI} has given birth to numerous
applications that have pushed the boundaries of what we previously believed to be possible.
This thesis will delve into one of the most fascinating and alarming developments in this
field, deepfakes. This document seeks to provide an exhaustive review of the current state
of the art in publicly-available deepfake detection tools.



\section{Background and Motivation}\label{background:backgroundAndMotivation}
In an era where digital media forms the cornerstone of communication, the advent of deepfakes,
\ac{AI}-enabled synthetic media, poses an unprecedented challenge to information integrity.
Deepfakes, a portmanteau of `deep learning' and `fake', is a technology that manipulates or
fabricates audio-visual content to make it appear real, often indistinguishable from the original.

The proliferation of deepfake technology was initially sparked by its application in creating
misleading celebrity images and videos, before quickly expanding into other sectors. One of the
earliest examples that drew significant attention to deepfakes was a video created by an anonymous
Reddit user called `deepfakes' in late 2017. This user began to post digitally altered pornographic
videos, realistically swapping the faces of actresses onto the bodies of porn stars. However, it
wasn't long before the technology was used outside of pornographic content.

A notable instance that clearly demonstrated the power of deepfakes, and arguably brought it to
mainstream attention, was a video of former U.S. President Barack Obama, released in April 2018
by Buzzfeed and Jordan Peele~\cite{peele,10.1145/3371409}. The video features a deepfake of Obama 
saying things he never actually said, with Peele providing the voiceover. This deepfake video, 
viewed by millions, effectively highlighted the potential misuse of this technology in spreading 
misinformation and propaganda.

In recent years, the sophistication of deepfake technology has reached an unprecedented level. 
A perfect example of this progression can be seen in the creation of `Tom Cruise deepfakes' 
that circulated on social media in early 2021. The videos, created by Belgian visual effects 
artist Chris Ume in collaboration with actor Miles Fisher, who impersonated Cruise's voice 
and mannerisms, were shared on TikTok under the account name @deeptomcruise. These deepfake 
videos show the synthetic `Tom Cruise' doing various activities - performing a magic trick, 
playing golf, or simply telling a story about Mikhail Gorbachev.

The `Tom Cruise deepfakes' took the internet by storm due to their uncanny resemblance to the 
real actor, in terms of both appearance and behavior. Unlike the early deepfake videos, which 
often exhibited glaring imperfections, these deepfakes were so convincing that many viewers 
initially believed they were watching the actual Tom Cruise. This level of realism underscored 
the strides made in deepfake technology, while simultaneously highlighting the potential 
dangers of its misuse.

Driven by advances in machine learning, especially deep learning, deepfake technology has
grown significantly in sophistication and accessibility. The potential applications of
deepfakes range from benign, such as in film production and entertainment, to malicious uses,
including disinformation campaigns, identity theft, and deepfake pornography. As these
applications become more widespread, deepfake technology has raised profound questions and
challenges for society, especially regarding media authenticity, privacy, and cybersecurity.

However, it is not just the creation of deepfakes that has improved; strides have also been made 
in detection. There are now more sophisticated, \ac{AI}-powered tools that can analyze videos and 
images for signs of manipulation. These tools operate on multiple levels, from detecting 
inconsistencies in lighting and shadows to looking for signs of digital artifacts and abnormal 
facial movements. But as detection tools become more sophisticated, so too do the techniques 
used to create deepfakes. This constantly evolving technological arms race underscores the 
critical need for ongoing research and development in deepfake detection.

In response to these challenges, there is an increasing need for robust and reliable
deepfake detection tools. However, despite the flurry of research and development in this
area, a comprehensive understanding and evaluation of the available detection tools remain
elusive. This knowledge gap not only impedes the technological advancements in deepfake
detection but also complicates the task of policy-making and regulation in this sphere.

This thesis is motivated by the need to bridge this gap and advance our understanding of
publicly-available deepfake detection tools. By examining these tools, this study aims to
contribute to the ongoing efforts to mitigate the risks associated with deepfakes and uphold
the integrity of digital media.


\section{Objectives of the Study}\label{objective:objectives}
\section{Scope and Limitations}\label{scope:scope}
\section{Thesis Structure}\label{thesis:thesis}
