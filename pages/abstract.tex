\chapter{\abstractname}

Deepfake technology, a fusion of deep learning and fake media, has rapidly 
evolved and become a powerful tool for generating highly realistic synthetic 
content. This advancement brings with it significant challenges in media 
authentication, entertainment industry, and privacy. As deepfakes become more 
sophisticated and accessible, the need for effective detection tools has become 
paramount. This thesis aims to provide a comprehensive understanding of 
the state of the art of publicly-available deepfake detection tools.

The study begins with a literature review that explores the evolution of 
deepfake technology, the various methods used for deepfake generation, 
and the existing approaches for deepfake detection. 

A solid methodology is used to collect and study data on the existing tools. 
They are evaluated based on factors like precision, speed, accessibility, 
and ease of use. The selected deepfake detection tools are assessed in detail 
to provide insights into their features, capabilities, and performance.

The findings of this study highlights the pros and cons of the tested 
deepfake detection methods. By comparing them, we understand their unique features 
and how well they identify deepfakes in various media. The research also points out 
current issues in deepfake detection and suggests directions for upcoming studies.

This research has consequences across various areas such as media, entertainment, 
and legal matters. Recognizing the difference between real and manipulated content 
is vital for protecting the integrity of information, preserving trust, and 
fighting against false information. The knowledge shared in this research contribute 
to the ongoing efforts to develop effective deepfake detection mechanisms.

In conclusion, this thesis provides a comprehensive overview of publicly-available 
deepfake detection tools, offering a thorough evaluation and comparison of their 
features and capabilities. The study highlights the need for ongoing research and 
development in the field of deepfake detection to counter the growing threat 
posed by synthetic media. By promoting a deeper understanding of the state of 
the art in deepfake detection, this research aims to contribute to the 
advancement of techniques that can effectively mitigate the risks associated 
with deepfakes and synthetic media.
