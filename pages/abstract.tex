\chapter{\abstractname}

Deepfake technology, a fusion of deep learning and fake media, has rapidly
evolved and become a powerful tool for generating highly realistic synthetic
content. This evolution poses challenges in media authentication, entertainment,
and privacy. As deepfakes become more sophisticated, the demand for effective
detection tools rises. This thesis seeks to offer an overview of publicly-available
deepfake detection tools.

Starting with a literature review, the research delves into the progression of
deepfake technology, its generation methods, and prevailing detection strategies.
A solid methodology with research questions was employed to gather and analyze
data on existing tools, evaluating them based on precision, accessibility, privacy
policies and user-friendliness.

The findings of this study highlight the pros and cons of the tested
deepfake detection methods. By comparing them, we understand their unique features
and how well they identify deepfakes in various media. The research revealed that
while some tools effectively identified video manipulations, they often struggled
with image deepfakes. Tools lacking regular updates, such as Facetorch, found it
challenging to detect newer deepfake techniques. Additionally, not all tools
provided clear privacy policies, emphasizing the need for user data transparency.

This research has consequences across various areas such as media, entertainment,
and legal matters. Recognizing the difference between real and manipulated content
is vital for protecting the integrity of information, preserving trust, and
fighting against false information. The knowledge shared in this research contributes
to the ongoing efforts to develop effective deepfake detection mechanisms.

In conclusion, this thesis provides a general overview of publicly-available
deepfake detection tools, offering an evaluation and comparison of their
features and capabilities. The study highlights the need for ongoing research and
development in the field of deepfake detection to counter the growing threat
posed by synthetic media. By promoting a deeper understanding of the state of
the art in deepfake detection, this research aims to contribute to the
advancement of techniques that can effectively mitigate the risks associated
with deepfakes and synthetic media.
