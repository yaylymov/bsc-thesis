\chapter{\abstractname}

Deepfake technology, a fusion of deep learning and fake media, has rapidly evolved and become a powerful tool for generating 
highly realistic synthetic content. This advancement brings with it significant challenges in media authentication, 
cybersecurity, and privacy. As deepfakes become more sophisticated and accessible, the need for effective detection tools 
has become paramount. This thesis aims to provide a comprehensive understanding of the state of the art of publicly-available 
deepfake detection tools.

The study begins with a literature review that explores the evolution of deepfake technology, the various methods used for deepfake 
generation, and the existing approaches for deepfake detection. By analyzing the strengths and limitations of these techniques, this 
study sets the foundation for evaluating the effectiveness of publicly-available deepfake detection tools.

A robust methodology is employed to collect and analyze data on the available tools. The evaluation criteria include accuracy, 
efficiency, scalability, versatility, and user-friendliness. The selected deepfake detection tools, encompassing open-source projects, 
commercial offerings, and academic research projects, are assessed in detail to provide insights into their features, capabilities, and performance.

The findings of this study reveal the strengths and weaknesses of the evaluated deepfake detection tools. Comparative analysis sheds 
light on their distinctive characteristics and effectiveness in detecting deepfakes across different media types. Additionally, the study 
identifies gaps and challenges within the current landscape of deepfake detection, offering recommendations for future research, 
development, and policy-making.

The implications of this research extend to a wide range of domains, including media forensics, journalism, law enforcement, and 
online platforms. The ability to distinguish between genuine and manipulated content is crucial for safeguarding information integrity, 
maintaining trust, and combating disinformation campaigns. The insights provided by this thesis contribute to the ongoing efforts to develop 
effective deepfake detection mechanisms that keep pace with the evolving landscape of deepfake technology.

In conclusion, this thesis provides a comprehensive overview of publicly-available deepfake detection tools, offering an in-depth evaluation 
and comparison of their features and capabilities. The study highlights the urgent need for ongoing research and development in the field of 
deepfake detection to counter the growing threat posed by synthetic media manipulation. By promoting a deeper understanding of the state of 
the art in deepfake detection, this research aims to contribute to the advancement of techniques and policies that can effectively mitigate 
the risks associated with deepfakes and uphold the integrity of digital media.
